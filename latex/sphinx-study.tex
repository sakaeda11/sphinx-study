%% Generated by Sphinx.
\def\sphinxdocclass{jsbook}
\documentclass[letterpaper,10pt,dvipdfmx,openany,oneside]{sphinxmanual}
\ifdefined\pdfpxdimen
   \let\sphinxpxdimen\pdfpxdimen\else\newdimen\sphinxpxdimen
\fi \sphinxpxdimen=.75bp\relax



\usepackage{cmap}
\usepackage[T1]{fontenc}
\usepackage{amsmath,amssymb,amstext}

\usepackage{times}

\usepackage[dontkeepoldnames]{sphinx}

\usepackage[dvipdfm]{geometry}

% Include hyperref last.
\usepackage{hyperref}
% Fix anchor placement for figures with captions.
\usepackage{hypcap}% it must be loaded after hyperref.
% Set up styles of URL: it should be placed after hyperref.
\urlstyle{same}
\renewcommand{\contentsname}{目次}

\renewcommand{\figurename}{図}
\renewcommand{\tablename}{表}
\renewcommand{\literalblockname}{リスト}

\renewcommand{\literalblockcontinuedname}{continued from previous page}
\renewcommand{\literalblockcontinuesname}{continues on next page}

\def\pageautorefname{ページ}

\setcounter{tocdepth}{1}



\title{sphinx-study}
\date{2020年12月28日}
\release{1.0.0}
\author{}
\newcommand{\sphinxlogo}{\vbox{}}
\renewcommand{\releasename}{リリース}
\makeindex

\begin{document}

\maketitle
\sphinxtableofcontents
\phantomsection\label{\detokenize{index::doc}}


このドキュメントは、ドキュメント生成ツールSphinxのサンプルです。
\begin{itemize}
\item {} \begin{description}
\item[{詳しくは以下のQiita記事やGithubレポジトリを参照して下さい}] \leavevmode\begin{itemize}
\item {} 
\sphinxhref{https://qiita.com/sakaeda11/items/29b96daa58498656e9b5}{Qiita: Sphinx \& reStructuredText入門}

\item {} 
\sphinxhref{https://github.com/sakaeda11/sphinx-study}{Github: sakaeda11/sphinx-study}

\end{itemize}

\end{description}

\end{itemize}


\chapter{This page is about Basic syntax}
\label{\detokenize{1.chapter/basic_syntax:this-page-is-about-basic-syntax}}\label{\detokenize{1.chapter/basic_syntax::doc}}\label{\detokenize{1.chapter/basic_syntax:sphinx-sample-project}}

\section{Sections}
\label{\detokenize{1.chapter/basic_syntax:sections}}
見出し文字の下に以下の記号文字のいずれかを**見出し文字の数より多く**並べる(アンダーラインとして並べる)ことで、ドキュメントの各セクション(見出し)を作成できます。

\fvset{hllines={, ,}}%
\begin{sphinxVerbatim}[commandchars=\\\{\}]
! \PYGZdq{} \PYGZsh{} \PYGZdl{} \PYGZpc{} \PYGZam{} \PYGZsq{} ( ) * + , \PYGZhy{} . / : ; \PYGZlt{} = \PYGZgt{} ? @ [ \PYGZbs{} ] \PYGZca{} \PYGZus{} {}` \PYGZob{} \textbar{} \PYGZcb{} \PYGZti{}
\end{sphinxVerbatim}

これらの内特に以下を利用することが推奨されます。

\fvset{hllines={, ,}}%
\begin{sphinxVerbatim}[commandchars=\\\{\}]
= \PYGZhy{} {}` : . \PYGZsq{} \PYGZdq{} \PYGZti{} \PYGZca{} \PYGZus{} * + \PYGZsh{}
\end{sphinxVerbatim}

例:

\fvset{hllines={, ,}}%
\begin{sphinxVerbatim}[commandchars=\\\{\}]
\PYG{c+c1}{\PYGZsh{}\PYGZsh{}\PYGZsh{}\PYGZsh{}\PYGZsh{}\PYGZsh{}\PYGZsh{}\PYGZsh{}\PYGZsh{}}
\PYG{n}{大見出し}
\PYG{c+c1}{\PYGZsh{}\PYGZsh{}\PYGZsh{}\PYGZsh{}\PYGZsh{}\PYGZsh{}\PYGZsh{}\PYGZsh{}\PYGZsh{}}

\PYG{n}{本文}

\PYG{o}{*}\PYG{o}{*}\PYG{o}{*}\PYG{o}{*}\PYG{o}{*}\PYG{o}{*}\PYG{o}{*}\PYG{o}{*}\PYG{o}{*}\PYG{o}{*}
\PYG{n}{中見出し}
\PYG{o}{*}\PYG{o}{*}\PYG{o}{*}\PYG{o}{*}\PYG{o}{*}\PYG{o}{*}\PYG{o}{*}\PYG{o}{*}\PYG{o}{*}\PYG{o}{*}

\PYG{n}{本文}

\PYG{o}{==}\PYG{o}{==}\PYG{o}{==}\PYG{o}{==}
\PYG{n}{小見出し}
\PYG{o}{==}\PYG{o}{==}\PYG{o}{==}\PYG{o}{==}

\PYG{n}{本文}
\end{sphinxVerbatim}

↓

\noindent\sphinxincludegraphics{{section-sample}.png}

\fvset{hllines={, ,}}%
\begin{sphinxVerbatim}[commandchars=\\\{\}]
注意:
・ 見栄えを良くするためにオーバーラインを書くこともできます。(見出しの上に並べる事ができる)
・ オーバーラインを記述する場合はアンダーラインと同じ文字数で書く必要があるようです。
・  Section用文字は以下の順序で利用することが推奨されているようです
  1. \PYGZsh{}
  2. *
  3. =
  4. \PYGZhy{}
  5. \PYGZca{}
  6. \PYGZdq{}
\end{sphinxVerbatim}


\section{List}
\label{\detokenize{1.chapter/basic_syntax:list}}
\fvset{hllines={, ,}}%
\begin{sphinxVerbatim}[commandchars=\\\{\}]
\PYG{o}{*} \PYG{n}{This} \PYG{o+ow}{is} \PYG{n}{a} \PYG{n}{bulleted} \PYG{n+nb}{list}\PYG{o}{.}
\PYG{o}{*} \PYG{n}{It} \PYG{n}{has} \PYG{n}{two} \PYG{n}{items}\PYG{p}{,} \PYG{n}{the} \PYG{n}{second}
\PYG{n}{item} \PYG{n}{uses} \PYG{n}{two} \PYG{n}{lines}\PYG{o}{.}

\PYG{l+m+mf}{1.} \PYG{n}{This} \PYG{o+ow}{is} \PYG{n}{a} \PYG{n}{numbered} \PYG{n+nb}{list}\PYG{o}{.}
\PYG{l+m+mf}{2.} \PYG{n}{It} \PYG{n}{has} \PYG{n}{two} \PYG{n}{items} \PYG{n}{too}\PYG{o}{.}

\PYG{c+c1}{\PYGZsh{}. This is a numbered list.}
\PYG{c+c1}{\PYGZsh{}. It has two items too.}
\end{sphinxVerbatim}

↓
\begin{itemize}
\item {} 
This is a bulleted list.

\item {} 
It has two items, the second
item uses two lines.

\end{itemize}
\begin{enumerate}
\item {} 
This is a numbered list.

\item {} 
It has two items too.

\item {} 
This is a numbered list.

\item {} 
It has two items too.

\end{enumerate}


\section{Literal blocks}
\label{\detokenize{1.chapter/basic_syntax:literal-blocks}}
\fvset{hllines={, ,}}%
\begin{sphinxVerbatim}[commandchars=\\\{\}]
\PYG{n}{This} \PYG{o+ow}{is} \PYG{n}{a} \PYG{n}{normal} \PYG{n}{text} \PYG{n}{paragraph}\PYG{o}{.} \PYG{n}{The} \PYG{n+nb}{next} \PYG{n}{paragraph} \PYG{o+ow}{is} \PYG{n}{a} \PYG{n}{code} \PYG{n}{sample}\PYG{p}{:}\PYG{p}{:}

   \PYG{n}{It} \PYG{o+ow}{is} \PYG{o+ow}{not} \PYG{n}{processed} \PYG{o+ow}{in} \PYG{n+nb}{any} \PYG{n}{way}\PYG{p}{,} \PYG{k}{except}
   \PYG{n}{that} \PYG{n}{the} \PYG{n}{indentation} \PYG{o+ow}{is} \PYG{n}{removed}\PYG{o}{.}

   \PYG{n}{It} \PYG{n}{can} \PYG{n}{span} \PYG{n}{multiple} \PYG{n}{lines}\PYG{o}{.}

\PYG{n}{This} \PYG{o+ow}{is} \PYG{n}{a} \PYG{n}{normal} \PYG{n}{text} \PYG{n}{paragraph} \PYG{n}{again}\PYG{o}{.}
\end{sphinxVerbatim}

↓

This is a normal text paragraph. The next paragraph is a code sample:

\fvset{hllines={, ,}}%
\begin{sphinxVerbatim}[commandchars=\\\{\}]
\PYG{n}{It} \PYG{o+ow}{is} \PYG{o+ow}{not} \PYG{n}{processed} \PYG{o+ow}{in} \PYG{n+nb}{any} \PYG{n}{way}\PYG{p}{,} \PYG{k}{except}
\PYG{n}{that} \PYG{n}{the} \PYG{n}{indentation} \PYG{o+ow}{is} \PYG{n}{removed}\PYG{o}{.}

\PYG{n}{It} \PYG{n}{can} \PYG{n}{span} \PYG{n}{multiple} \PYG{n}{lines}\PYG{o}{.}
\end{sphinxVerbatim}

This is a normal text paragraph again.


\section{Doctest blocks}
\label{\detokenize{1.chapter/basic_syntax:doctest-blocks}}
\fvset{hllines={, ,}}%
\begin{sphinxVerbatim}[commandchars=\\\{\}]
\PYG{g+gp}{\PYGZgt{}\PYGZgt{}\PYGZgt{} }\PYG{l+m+mi}{1} \PYG{o}{+} \PYG{l+m+mi}{1}
\PYG{g+go}{2}
\end{sphinxVerbatim}

のように \sphinxstylestrong{\textgreater{}\textgreater{}\textgreater{}} と記述するとデコレーションされます

↓

\fvset{hllines={, ,}}%
\begin{sphinxVerbatim}[commandchars=\\\{\}]
\PYG{g+gp}{\PYGZgt{}\PYGZgt{}\PYGZgt{} }\PYG{l+m+mi}{1} \PYG{o}{+} \PYG{l+m+mi}{1}
\PYG{g+go}{2}
\end{sphinxVerbatim}


\section{Tables}
\label{\detokenize{1.chapter/basic_syntax:tables}}
テーブルの表記方法にはいくつかあります。


\subsection{グリッドテーブル表記}
\label{\detokenize{1.chapter/basic_syntax:id1}}
\fvset{hllines={, ,}}%
\begin{sphinxVerbatim}[commandchars=\\\{\}]
\PYG{o}{+}\PYG{o}{\PYGZhy{}}\PYG{o}{\PYGZhy{}}\PYG{o}{\PYGZhy{}}\PYG{o}{\PYGZhy{}}\PYG{o}{\PYGZhy{}}\PYG{o}{\PYGZhy{}}\PYG{o}{\PYGZhy{}}\PYG{o}{\PYGZhy{}}\PYG{o}{\PYGZhy{}}\PYG{o}{\PYGZhy{}}\PYG{o}{\PYGZhy{}}\PYG{o}{\PYGZhy{}}\PYG{o}{\PYGZhy{}}\PYG{o}{\PYGZhy{}}\PYG{o}{\PYGZhy{}}\PYG{o}{\PYGZhy{}}\PYG{o}{\PYGZhy{}}\PYG{o}{\PYGZhy{}}\PYG{o}{\PYGZhy{}}\PYG{o}{\PYGZhy{}}\PYG{o}{\PYGZhy{}}\PYG{o}{\PYGZhy{}}\PYG{o}{\PYGZhy{}}\PYG{o}{\PYGZhy{}}\PYG{o}{+}\PYG{o}{\PYGZhy{}}\PYG{o}{\PYGZhy{}}\PYG{o}{\PYGZhy{}}\PYG{o}{\PYGZhy{}}\PYG{o}{\PYGZhy{}}\PYG{o}{\PYGZhy{}}\PYG{o}{\PYGZhy{}}\PYG{o}{\PYGZhy{}}\PYG{o}{\PYGZhy{}}\PYG{o}{\PYGZhy{}}\PYG{o}{\PYGZhy{}}\PYG{o}{\PYGZhy{}}\PYG{o}{+}\PYG{o}{\PYGZhy{}}\PYG{o}{\PYGZhy{}}\PYG{o}{\PYGZhy{}}\PYG{o}{\PYGZhy{}}\PYG{o}{\PYGZhy{}}\PYG{o}{\PYGZhy{}}\PYG{o}{\PYGZhy{}}\PYG{o}{\PYGZhy{}}\PYG{o}{\PYGZhy{}}\PYG{o}{\PYGZhy{}}\PYG{o}{+}\PYG{o}{\PYGZhy{}}\PYG{o}{\PYGZhy{}}\PYG{o}{\PYGZhy{}}\PYG{o}{\PYGZhy{}}\PYG{o}{\PYGZhy{}}\PYG{o}{\PYGZhy{}}\PYG{o}{\PYGZhy{}}\PYG{o}{\PYGZhy{}}\PYG{o}{\PYGZhy{}}\PYG{o}{\PYGZhy{}}\PYG{o}{+}
\PYG{o}{\textbar{}} \PYG{n}{Header} \PYG{n}{row}\PYG{p}{,} \PYG{n}{column} \PYG{l+m+mi}{1}   \PYG{o}{\textbar{}} \PYG{n}{Header} \PYG{l+m+mi}{2}   \PYG{o}{\textbar{}} \PYG{n}{Header} \PYG{l+m+mi}{3} \PYG{o}{\textbar{}} \PYG{n}{Header} \PYG{l+m+mi}{4} \PYG{o}{\textbar{}}
\PYG{o}{\textbar{}} \PYG{p}{(}\PYG{n}{header} \PYG{n}{rows} \PYG{n}{optional}\PYG{p}{)} \PYG{o}{\textbar{}}            \PYG{o}{\textbar{}}          \PYG{o}{\textbar{}}          \PYG{o}{\textbar{}}
\PYG{o}{+}\PYG{o}{==}\PYG{o}{==}\PYG{o}{==}\PYG{o}{==}\PYG{o}{==}\PYG{o}{==}\PYG{o}{==}\PYG{o}{==}\PYG{o}{==}\PYG{o}{==}\PYG{o}{==}\PYG{o}{==}\PYG{o}{+}\PYG{o}{==}\PYG{o}{==}\PYG{o}{==}\PYG{o}{==}\PYG{o}{==}\PYG{o}{==}\PYG{o}{+}\PYG{o}{==}\PYG{o}{==}\PYG{o}{==}\PYG{o}{==}\PYG{o}{==}\PYG{o}{+}\PYG{o}{==}\PYG{o}{==}\PYG{o}{==}\PYG{o}{==}\PYG{o}{==}\PYG{o}{+}
\PYG{o}{\textbar{}} \PYG{n}{body} \PYG{n}{row} \PYG{l+m+mi}{1}\PYG{p}{,} \PYG{n}{column} \PYG{l+m+mi}{1}   \PYG{o}{\textbar{}} \PYG{n}{column} \PYG{l+m+mi}{2}   \PYG{o}{\textbar{}} \PYG{n}{column} \PYG{l+m+mi}{3} \PYG{o}{\textbar{}} \PYG{n}{column} \PYG{l+m+mi}{4} \PYG{o}{\textbar{}}
\PYG{o}{+}\PYG{o}{\PYGZhy{}}\PYG{o}{\PYGZhy{}}\PYG{o}{\PYGZhy{}}\PYG{o}{\PYGZhy{}}\PYG{o}{\PYGZhy{}}\PYG{o}{\PYGZhy{}}\PYG{o}{\PYGZhy{}}\PYG{o}{\PYGZhy{}}\PYG{o}{\PYGZhy{}}\PYG{o}{\PYGZhy{}}\PYG{o}{\PYGZhy{}}\PYG{o}{\PYGZhy{}}\PYG{o}{\PYGZhy{}}\PYG{o}{\PYGZhy{}}\PYG{o}{\PYGZhy{}}\PYG{o}{\PYGZhy{}}\PYG{o}{\PYGZhy{}}\PYG{o}{\PYGZhy{}}\PYG{o}{\PYGZhy{}}\PYG{o}{\PYGZhy{}}\PYG{o}{\PYGZhy{}}\PYG{o}{\PYGZhy{}}\PYG{o}{\PYGZhy{}}\PYG{o}{\PYGZhy{}}\PYG{o}{+}\PYG{o}{\PYGZhy{}}\PYG{o}{\PYGZhy{}}\PYG{o}{\PYGZhy{}}\PYG{o}{\PYGZhy{}}\PYG{o}{\PYGZhy{}}\PYG{o}{\PYGZhy{}}\PYG{o}{\PYGZhy{}}\PYG{o}{\PYGZhy{}}\PYG{o}{\PYGZhy{}}\PYG{o}{\PYGZhy{}}\PYG{o}{\PYGZhy{}}\PYG{o}{\PYGZhy{}}\PYG{o}{+}\PYG{o}{\PYGZhy{}}\PYG{o}{\PYGZhy{}}\PYG{o}{\PYGZhy{}}\PYG{o}{\PYGZhy{}}\PYG{o}{\PYGZhy{}}\PYG{o}{\PYGZhy{}}\PYG{o}{\PYGZhy{}}\PYG{o}{\PYGZhy{}}\PYG{o}{\PYGZhy{}}\PYG{o}{\PYGZhy{}}\PYG{o}{+}\PYG{o}{\PYGZhy{}}\PYG{o}{\PYGZhy{}}\PYG{o}{\PYGZhy{}}\PYG{o}{\PYGZhy{}}\PYG{o}{\PYGZhy{}}\PYG{o}{\PYGZhy{}}\PYG{o}{\PYGZhy{}}\PYG{o}{\PYGZhy{}}\PYG{o}{\PYGZhy{}}\PYG{o}{\PYGZhy{}}\PYG{o}{+}
\PYG{o}{\textbar{}} \PYG{n}{body} \PYG{n}{row} \PYG{l+m+mi}{2}             \PYG{o}{\textbar{}} \PYG{o}{.}\PYG{o}{.}\PYG{o}{.}        \PYG{o}{\textbar{}} \PYG{o}{.}\PYG{o}{.}\PYG{o}{.}      \PYG{o}{\textbar{}}          \PYG{o}{\textbar{}}
\PYG{o}{+}\PYG{o}{\PYGZhy{}}\PYG{o}{\PYGZhy{}}\PYG{o}{\PYGZhy{}}\PYG{o}{\PYGZhy{}}\PYG{o}{\PYGZhy{}}\PYG{o}{\PYGZhy{}}\PYG{o}{\PYGZhy{}}\PYG{o}{\PYGZhy{}}\PYG{o}{\PYGZhy{}}\PYG{o}{\PYGZhy{}}\PYG{o}{\PYGZhy{}}\PYG{o}{\PYGZhy{}}\PYG{o}{\PYGZhy{}}\PYG{o}{\PYGZhy{}}\PYG{o}{\PYGZhy{}}\PYG{o}{\PYGZhy{}}\PYG{o}{\PYGZhy{}}\PYG{o}{\PYGZhy{}}\PYG{o}{\PYGZhy{}}\PYG{o}{\PYGZhy{}}\PYG{o}{\PYGZhy{}}\PYG{o}{\PYGZhy{}}\PYG{o}{\PYGZhy{}}\PYG{o}{\PYGZhy{}}\PYG{o}{+}\PYG{o}{\PYGZhy{}}\PYG{o}{\PYGZhy{}}\PYG{o}{\PYGZhy{}}\PYG{o}{\PYGZhy{}}\PYG{o}{\PYGZhy{}}\PYG{o}{\PYGZhy{}}\PYG{o}{\PYGZhy{}}\PYG{o}{\PYGZhy{}}\PYG{o}{\PYGZhy{}}\PYG{o}{\PYGZhy{}}\PYG{o}{\PYGZhy{}}\PYG{o}{\PYGZhy{}}\PYG{o}{+}\PYG{o}{\PYGZhy{}}\PYG{o}{\PYGZhy{}}\PYG{o}{\PYGZhy{}}\PYG{o}{\PYGZhy{}}\PYG{o}{\PYGZhy{}}\PYG{o}{\PYGZhy{}}\PYG{o}{\PYGZhy{}}\PYG{o}{\PYGZhy{}}\PYG{o}{\PYGZhy{}}\PYG{o}{\PYGZhy{}}\PYG{o}{+}\PYG{o}{\PYGZhy{}}\PYG{o}{\PYGZhy{}}\PYG{o}{\PYGZhy{}}\PYG{o}{\PYGZhy{}}\PYG{o}{\PYGZhy{}}\PYG{o}{\PYGZhy{}}\PYG{o}{\PYGZhy{}}\PYG{o}{\PYGZhy{}}\PYG{o}{\PYGZhy{}}\PYG{o}{\PYGZhy{}}\PYG{o}{+}
\end{sphinxVerbatim}

↓


\begin{savenotes}\sphinxattablestart
\centering
\begin{tabulary}{\linewidth}[t]{|T|T|T|T|}
\hline
\sphinxstylethead{\sphinxstyletheadfamily 
Header row, column 1
(header rows optional)
\unskip}\relax &\sphinxstylethead{\sphinxstyletheadfamily 
Header 2
\unskip}\relax &\sphinxstylethead{\sphinxstyletheadfamily 
Header 3
\unskip}\relax &\sphinxstylethead{\sphinxstyletheadfamily 
Header 4
\unskip}\relax \\
\hline
body row 1, column 1
&
column 2
&
column 3
&
column 4
\\
\hline
body row 2
&
...
&
...
&\\
\hline
\end{tabulary}
\par
\sphinxattableend\end{savenotes}


\subsection{シンプルテーブル表記}
\label{\detokenize{1.chapter/basic_syntax:id2}}
\fvset{hllines={, ,}}%
\begin{sphinxVerbatim}[commandchars=\\\{\}]
\PYG{o}{==}\PYG{o}{==}\PYG{o}{=}  \PYG{o}{==}\PYG{o}{==}\PYG{o}{=}  \PYG{o}{==}\PYG{o}{==}\PYG{o}{==}\PYG{o}{=}
\PYG{n}{A}      \PYG{n}{B}      \PYG{n}{A} \PYG{o+ow}{and} \PYG{n}{B}
\PYG{o}{==}\PYG{o}{==}\PYG{o}{=}  \PYG{o}{==}\PYG{o}{==}\PYG{o}{=}  \PYG{o}{==}\PYG{o}{==}\PYG{o}{==}\PYG{o}{=}
\PYG{k+kc}{False}  \PYG{k+kc}{False}  \PYG{k+kc}{False}
\PYG{k+kc}{True}   \PYG{k+kc}{False}  \PYG{k+kc}{False}
\PYG{k+kc}{False}  \PYG{k+kc}{True}   \PYG{k+kc}{False}
\PYG{k+kc}{True}   \PYG{k+kc}{True}   \PYG{k+kc}{True}
\PYG{o}{==}\PYG{o}{==}\PYG{o}{=}  \PYG{o}{==}\PYG{o}{==}\PYG{o}{=}  \PYG{o}{==}\PYG{o}{==}\PYG{o}{==}\PYG{o}{=}
\end{sphinxVerbatim}

↓


\begin{savenotes}\sphinxattablestart
\centering
\begin{tabulary}{\linewidth}[t]{|T|T|T|}
\hline
\sphinxstylethead{\sphinxstyletheadfamily 
A
\unskip}\relax &\sphinxstylethead{\sphinxstyletheadfamily 
B
\unskip}\relax &\sphinxstylethead{\sphinxstyletheadfamily 
A and B
\unskip}\relax \\
\hline
False
&
False
&
False
\\
\hline
True
&
False
&
False
\\
\hline
False
&
True
&
False
\\
\hline
True
&
True
&
True
\\
\hline
\end{tabulary}
\par
\sphinxattableend\end{savenotes}


\subsection{csv-tableディレクティブ表記}
\label{\detokenize{1.chapter/basic_syntax:csv-table}}
\fvset{hllines={, ,}}%
\begin{sphinxVerbatim}[commandchars=\\\{\}]
\PYG{o}{.}\PYG{o}{.} \PYG{n}{csv}\PYG{o}{\PYGZhy{}}\PYG{n}{table}\PYG{p}{:}\PYG{p}{:}
   \PYG{p}{:}\PYG{n}{header}\PYG{p}{:} \PYG{l+s+s2}{\PYGZdq{}}\PYG{l+s+s2}{AAA}\PYG{l+s+s2}{\PYGZdq{}}\PYG{p}{,} \PYG{l+s+s2}{\PYGZdq{}}\PYG{l+s+s2}{BBB}\PYG{l+s+s2}{\PYGZdq{}}\PYG{p}{,} \PYG{l+s+s2}{\PYGZdq{}}\PYG{l+s+s2}{CCC}\PYG{l+s+s2}{\PYGZdq{}}
   \PYG{p}{:}\PYG{n}{width}\PYG{p}{:} \PYG{l+m+mi}{80}\PYG{o}{\PYGZpc{}}
   \PYG{p}{:}\PYG{n}{align}\PYG{p}{:} \PYG{n}{left}

   \PYG{n}{a}\PYG{p}{,}\PYG{n}{b}\PYG{p}{,}\PYG{n}{c}
   \PYG{l+m+mi}{1}\PYG{p}{,}\PYG{l+m+mi}{2}\PYG{p}{,}\PYG{l+m+mi}{3}
\end{sphinxVerbatim}

↓


\subsection{list-tableディレクティブ表記}
\label{\detokenize{1.chapter/basic_syntax:list-table}}
\fvset{hllines={, ,}}%
\begin{sphinxVerbatim}[commandchars=\\\{\}]
.. list\PYGZhy{}table:: Frozen Delights!
   :widths: 15 10 30
   :header\PYGZhy{}rows: 1

   * \PYGZhy{} Treat
   \PYGZhy{} Quantity
   \PYGZhy{} Description
   * \PYGZhy{} Albatross
   \PYGZhy{} 2.99
   \PYGZhy{} On a stick!
   * \PYGZhy{} Crunchy Frog
   \PYGZhy{} 1.49
   \PYGZhy{} If we took the bones out, it wouldn\PYGZsq{}t be
      crunchy, now would it?
   * \PYGZhy{} Gannet Ripple
   \PYGZhy{} 1.99
   \PYGZhy{} On a stick!
\end{sphinxVerbatim}

↓


\begin{savenotes}\sphinxattablestart
\centering
\sphinxcapstartof{table}
\sphinxcaption{Frozen Delights!}\label{\detokenize{1.chapter/basic_syntax:id6}}
\sphinxaftercaption
\begin{tabular}[t]{|\X{15}{55}|\X{10}{55}|\X{30}{55}|}
\hline
\sphinxstylethead{\sphinxstyletheadfamily 
Treat
\unskip}\relax &\sphinxstylethead{\sphinxstyletheadfamily 
Quantity
\unskip}\relax &\sphinxstylethead{\sphinxstyletheadfamily 
Description
\unskip}\relax \\
\hline
Albatross
&
2.99
&
On a stick!
\\
\hline
Crunchy Frog
&
1.49
&
If we took the bones out, it wouldn't be
crunchy, now would it?
\\
\hline
Gannet Ripple
&
1.99
&
On a stick!
\\
\hline
\end{tabular}
\par
\sphinxattableend\end{savenotes}


\section{Hyperlinks}
\label{\detokenize{1.chapter/basic_syntax:hyperlinks}}
リンクの書き方は主に2つあります

\fvset{hllines={, ,}}%
\begin{sphinxVerbatim}[commandchars=\\\{\}]
This is a paragraph that contains {}`a link{}`\PYGZus{}.

.. \PYGZus{}a link: https://domain.invalid/
\end{sphinxVerbatim}

↓

This is a paragraph that contains \sphinxhref{https://domain.invalid/}{a link}.

\fvset{hllines={, ,}}%
\begin{sphinxVerbatim}[commandchars=\\\{\}]
{}`Link text \PYGZlt{}https://domain.invalid/\PYGZgt{}{}`\PYGZus{}
\end{sphinxVerbatim}

↓

\sphinxhref{https://domain.invalid/}{Link text}


\section{Images}
\label{\detokenize{1.chapter/basic_syntax:images}}
画像を利用したい場合はimageディレクティブを利用します。
画像の場所は、該当のrstファイルからの相対パスと、トップのソースディレクトリからの絶対パスのいずれも利用できます。

\fvset{hllines={, ,}}%
\begin{sphinxVerbatim}[commandchars=\\\{\}]
\PYG{o}{.}\PYG{o}{.} \PYG{n}{image}\PYG{p}{:}\PYG{p}{:} \PYG{n}{imgs}\PYG{o}{/}\PYG{n}{sample1}\PYG{o}{.}\PYG{n}{jpg}
   \PYG{p}{:}\PYG{n}{width}\PYG{p}{:} \PYG{l+m+mi}{50} \PYG{o}{\PYGZpc{}}
   \PYG{p}{:}\PYG{n}{align}\PYG{p}{:} \PYG{n}{left}
\end{sphinxVerbatim}

↓

\noindent{\sphinxincludegraphics[width=0.500\linewidth]{{sample1}.jpg}\hspace*{\fill}}


\section{Footnotes}
\label{\detokenize{1.chapter/basic_syntax:footnotes}}\begin{itemize}
\item {} 
{[}\#name{]}\_ と  .. {[}\#name{]} を書くことで脚注を記述できます

\item {} 
nameは省略しても良いです。({[}\#{]}\_だけでも可で、自動採番されます。)

\item {} 
\#ではなく数字をそのまま記述しても良いです。

\end{itemize}

\fvset{hllines={, ,}}%
\begin{sphinxVerbatim}[commandchars=\\\{\}]
\PYG{n}{Lorem} \PYG{n}{ipsum} \PYG{p}{[}\PYG{c+c1}{\PYGZsh{}f1]\PYGZus{} dolor sit amet ... [\PYGZsh{}f2]\PYGZus{}}

\PYG{o}{.}\PYG{o}{.} \PYG{p}{[}\PYG{c+c1}{\PYGZsh{}f1] Text of the first footnote.}
\PYG{o}{.}\PYG{o}{.} \PYG{p}{[}\PYG{c+c1}{\PYGZsh{}f2] Text of the second footnote.}
\end{sphinxVerbatim}

↓

Lorem ipsum %
\begin{footnote}[1]\sphinxAtStartFootnote
Text of the first footnote.
%
\end{footnote} dolor sit amet ... %
\begin{footnote}[2]\sphinxAtStartFootnote
Text of the second footnote.
%
\end{footnote}

\begin{sphinxadmonition}{note}{注釈:}
html生成時は \sphinxstylestrong{rubric} ディレクティブを書く事で脚注についてそれらしい整形がされますが、PDF生成時は変な感じになるので書かない方が良さそうです。
\end{sphinxadmonition}


\section{Citations}
\label{\detokenize{1.chapter/basic_syntax:citations}}\begin{itemize}
\item {} 
引用、参考文献の記述方法です。

\item {} 
書き方はFootnote(脚注)と似ています。

\item {} 
このように \phantomsection\label{\detokenize{1.chapter/basic_syntax:id5}}{\hyperref[\detokenize{1.chapter/basic_syntax:abc}]{\sphinxcrossref{{[}参考文献名ABC{]}}}} と、それに対応するディレクティブを以下のように書いておくと、自動的に参考文献ページが巻末に生成されます。

\end{itemize}

\fvset{hllines={, ,}}%
\begin{sphinxVerbatim}[commandchars=\\\{\}]
\PYG{o}{.}\PYG{o}{.} \PYG{p}{[}\PYG{n}{参考文献名ABC}\PYG{p}{]} \PYG{l+m+mi}{2020} \PYG{n}{CDE著} \PYG{n}{FGH文庫}
\end{sphinxVerbatim}

\begin{sphinxadmonition}{warning}{警告:}
PDF生成ではそれらしい感じになりますが、HTML生成での利用はおすすめしません。
\end{sphinxadmonition}


\chapter{This page is about Directive}
\label{\detokenize{2.chapter/directive:this-page-is-about-directive}}\label{\detokenize{2.chapter/directive::doc}}

\section{This is Sub page 1}
\label{\detokenize{2.chapter/subpage/sub1:this-is-sub-page-1}}\label{\detokenize{2.chapter/subpage/sub1::doc}}

\section{This is Sub page 1}
\label{\detokenize{2.chapter/subpage/sub2:this-is-sub-page-1}}\label{\detokenize{2.chapter/subpage/sub2::doc}}
\begin{sphinxadmonition}{note}{注釈:}
".. note::"というディレクティブを記述した場合このようになります。
\end{sphinxadmonition}

\begin{sphinxadmonition}{warning}{警告:}
".. warning::"というディレクティブを記述した場合このようになります。
\end{sphinxadmonition}

\begin{sphinxthebibliography}{参考文献名ABC}
\bibitem[参考文献名ABC]{\detokenize{_u53c2_u8003_u6587_u732e_u540dABC}}{\phantomsection\label{\detokenize{1.chapter/basic_syntax:abc}} 
2020 CDE著 FGH文庫
}
\end{sphinxthebibliography}



\renewcommand{\indexname}{索引}
\printindex
\end{document}